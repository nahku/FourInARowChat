%!TEX root = ../dokumentation.tex

\chapter{Einleitung}\label{cha:Einleitung}

\section{Aufgabenstellung und Ziel der Projektarbeit}\label{sec:Aufgabenstellung}
Die Aufgabe bestand in der Entwicklung einer Beispielanwendung unter Verwendung zweier Aspekte aus der Vorlesung. Wir haben uns für die Entwicklung eines 4-gewinnt-Spiels auf Basis der Chat-Anwendung entschieden. Dieses soll mittels Docker deploybar gemacht werden. Die Chat-Teilnehmer sollen mit einer Registrierungsmöglichkeit in einer MongoDB verwaltet werden, welche auch in einen Docker-Container verpackt werden soll. Die Entwicklung der Anwendung wird mit Node.js durchgeführt. Es soll mehreren Spielern das gleichzeitige Spielen ermöglichen.
\section{Aufbau der Projektarbeit}\label{sec:Aufbau}
Diese Projektarbeit gliedert sich in fünf Kapitel. Im ersten Kapitel wird eine Einleitung in die Aufgabenstellung und die geplante Anwendung gegeben. Das zweite Kapitel beinhaltet die notwendigen Grundlagen, auf denen die Anwendung basiert. Dabei werden die einzelnen Technologien und ihre Möglichkeiten kurz vorgestellt. Darauffolgend wird in Kapitel 3 das Konzept der Anwendung und ihrer Komponenten dargestellt. Dabei wird auch auf die Vernetzung zwischen zum Beispiel der Datenbank und dem Chat eingegangen. In Kapitel 4 werden dann einzelne Teile der Implementierung vorgestellt, die essenziell für die Funktion der Anwedung sind. Im letzten Kapitel wird noch einmal auf die Aufgabenstellung Rückbezug genommen und eine kritische Würdigung der erzielten Ergebnisse vorgenommen. Zusätzlich dazu wird ein Ausbilck auf mögliche Weiterentwicklungen gegeben.
\section{Wer hat was gemacht?}\label{sec:Aufteilung}
Das Konzept der Anwendung wurde gemeinsam entwickelt. Der Großteil des Quellcodes wurden nach der pair programming Arbeitstechnik erstellt, wobei Hanna vorrangig für die Kommunikation verantwortlich war und Nahku die Spiellogik behandelt hat.