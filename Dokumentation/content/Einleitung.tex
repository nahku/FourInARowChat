%!TEX root = ../dokumentation.tex

\chapter{Einleitung}\label{cha:Einleitung}

\section{Aufgabenstellung und Ziel der Projektarbeit}\label{sec:Aufgabenstellung}
%Autonomes Fahren setzt große Herausforderungen an die Fahrzeugsensorik und deren Datenverarbeitung. Zusätzlich zu den bisher für Fahrerassistenzsysteme verwendeten Kamera- und Radarsensoren werden in Zukunft auch \ac{LiDAR}-Sensoren eingesetzt. \ac{LiDAR} bezeichnet eine optische Methode zur Abstands- und Geschwindigkeitsmessung und wird aufgrund der hohen Messgenauigkeit, der hohen Anzahl an Messpunkten und des großen Sichtfelds immer öfter eingesetzt. Außerdem beruht das Messprinzip auf einem zu den üblichen Umgebungssensoren unterschiedlichen physikalischen Prinzip und stellt damit eine Ausfallsicherheit zur üblichen Sensorik dar.\\
Mit der Entwicklung und Einführung des autonomen Fahrens ergeben sich hohe Anforderungen an die Umfeldsensorik und die Datenverarbeitung und -analyse. Neben den bereits für Fahrassistenzssysteme verwendeten Kamera- und Radarsensoren bietet sich auch die \ac{LiDAR}-Sensorik für Funktionen in diesem Umfeld an. Ein \ac{LiDAR} ist ein Umfeldsensor, der mittels optischer Methoden eine Abstandsmessung durchführt. Das Messprinzip ist dem \ac{Radar} ähnlich und basiert auf der \ac{TOF}-Messung. \ac{LiDAR} bietet ein großes Sichtfeld, eine hohe Messgenauigkeit und eine hohe Auflösung \cite[vgl.][141]{Reif.2010}. 
Aufgrund des hohen Potenzials dieser Technik und einer möglichen großen praktischen Relevanz für die Zukunft stellen \ac{LiDAR}-Objekterkennungen ein sehr aktuelles Forschungsthema dar.
Mit dieser Projektarbeit soll das Konzept einer \ac{LiDAR}-basierten Objekterkennung erarbeitet werden und eine prototypische Implementierung in Matlab vorgenommen werden. Diese soll mit einem industriellen 360°-3D-\ac{LiDAR}-Sensor die Umgebung des Sensors analysieren und Objekte innerhalb der vom Sensor erzeugten Daten erkennen. Die Objekte sollen im Sensordatenbild visualisiert, aber nicht klassifiziert werden. 
Dabei sollen bereits bestehende Ansätze zur Objekterkennung und Objektunterscheidung recherchiert und deren zu Grunde liegenden Charakteristiken ausgearbeitet werden.\\
Der verwendete \ac{LiDAR}-Sensor ist stationär. Somit wird dieser während der Datenaufnahme nicht bewegt. Die Objekterkennung soll vorrangig für geschlossene Innenräume konzipiert werden. Eine Echtzeitfähigkeit ist nicht erforderlich, da die Sensordaten nach der Aufnahme gespeichert und dann analysiert werden. Die Funktion der Objekterkennung soll anhand eines Testszenarios überprüft und ausgewertet werden.

\section{Aufbau der Projektarbeit}\label{sec:Aufbau}
Die Projektarbeit gliedert sich in sechs Kapitel. Im ersten Kapitel wird die Aufgabenstellung und das Ziel der Projektarbeit dargestellt und darauf aufbauend das Vorgehen und der Aufbau der Projektarbeit beschrieben.\\
Anschließend wird in Kapitel \ref{cha:Grundlagen} ein Überblick über die, vor allem im Automobilbereich, verwendete Umfeldsensorik gegeben. Dabei wird herausgearbeitet, was einen \ac{LiDAR}-Sensor von anderen, in Serienfahrzeugen verwendeten, Sensoren unterscheidet und welche Verwendungsmöglichkeiten diese Messtechnik im Vergleich zu den anderen Messtechniken bietet. Anschließend werden bisherige Ansätze zur \ac{LiDAR}-Objekterkennung diskutiert und deren Vor- und Nachteile herausgestellt.
Darauf aufbauend wird in Kapitel \ref{cha:Konzept} ein eigenes Konzept für eine \ac{LiDAR}-Objekterkennung erstellt, dessen Implementierung im darauffolgenden Kapitel \ref{cha:Implementierung} behandelt wird.\\
Kapitel \ref{cha:Evaluation} beinhaltet die Evaluation der implementierten \ac{LiDAR}-Objekterkennung anhand eines Testszenarios. Dabei wird die Performance der Objekterkennung ausgewertet und es werden die Vorteile und Grenzen dieser analysiert.\\
Abschließend wird im letzten Kapitel unter Rückbezugnahme auf die im ersten Kapitel definierten Ziele das Fazit gezogen. Dabei wird sowohl auf die Aufgabenstellung und die Vorgehensweise als auch die Forschungsergebnisse eingegangen. Zusätzlich wird ein Ausblick auf mögliche auf dieser Arbeit aufbauende Forschungen gegeben.