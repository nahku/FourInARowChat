%!TEX root = ../dokumentation.tex

\chapter{Projektabschluss, Fazit \& Ausblick}\label{cha:Schlussbetrachtung}
\section{Fazit}\label{sec:Fazit}
Das Ziel dieser Projektarbeit war die Entwicklung eines 4-gewinnt-Spiels auf Basis der in der Vorlesung vorgestellten Chat-Anwendung. Dazu wurde eine Anwendung entwickelt, die aus einem Client, einem Server und einer Datenbank besteht. Um die Echtzeitfähigkeit, die sowohl für den Chat als auch das Spiel benötigt wird, zu gewährleisten, wurde ein Websocket eingesetzt. Außerdem wurde die Anwendung und die MongoDB in jeweils einen Docker-Container verpackt. Die Mehrbenutzerfähigkeit wurde durch eine Verwaltung mehrerer Spielfelder zugehörig zu den Spielern erreicht. Registrierung und Login werden ermöglicht, indem die Benutzerdaten in der Datenbank gespeichert werden. Dadurch können neue Nutzer hinzugefügt werden und es kann überprüft werden, ob ein Nutzer die richtigen Zugangsdaten eingegeben hat.\\
Somit erfüllt die Anwendung alle definierten Anforderungen, die in Kapitel \ref{cha:Einleitung} und Kapitel \ref{sec:Anforderungen} genannt wurden.

\section{Ausblick}\label{sec:Ausblick}
Ein Punkt, an dem die Benutzerverwaltung noch verbessert werden kann, ist, dass nicht vollständig erkannt wird, ob ein Benutzername schon existiert. Sobald der gewünschte Username (z.B. test123) Teil eines schon bestehendes Usernamens ist (z.B. test12345), wird nicht erkannt, dass die Nutzernamen unterschiedlich sind. Daher würde der Benutzername als schon vergeben klassifiziert werden. Die Registrierung wird also verweigert, obwohl der Username noch nicht existiert. Durch eine genauere Prüfung der bereits existierenden Usernamen könnte dies verhindert werden.
Bei der Prüfung, ob der Nutzername bereits vergeben ist, schickt die Datenbank alle Datensätze, die entweder gleich sind oder den eingegebenen Namen enthalten, an den Server. Diese Datensätze werden als JSON-Array gesendet. Der Server müsste für eine vollständige Prüfung, das JSON-Array parsen und den Namen mit den eingegeben Daten vergleichen und anhand dessen feststellen, ob diese identisch sind oder nicht.

Außerdem werden die Passwörter der Nutzer momentan als Klartext in der MongoDB gespeichert. Für die Zukunft wäre es denkbar, die Passwörter verschlüsselt zu speichern. Dazu könnte man beispielsweise einen aus dem Nutzer-Passwort berechneten Hash-Code in der Datenbank speichern. Dies würde sich aus Gründen des Datenschutzes anbieten.

Eine denkbare Erweiterung der Anwendung ist das Speichern des Spielstandes in der Datenbank. Zurzeit ist das Spiel nur lokal zwischengespeichert und wird gelöscht, sobald das Spiel beendet ist oder die Seite neu geladen wird. Durch das Speichern des Spielstandes und des Spielfeldes in der Datenbank könnte auch bei einem erneuten Login das alte Spiel weitergespielt werden.\\
Um die Anwendung intuitiver zu gestalten, können dem User beim Login und der Registrierung Bildschirmausgaben über den aktuellen Stand angezeigt werden. Diese Ausgaben könnten über eine fehlgeschlagene Anmeldung informieren. Aktuell wird bei einer fehlerhaften Anmeldung die Seite erneut geladen. 