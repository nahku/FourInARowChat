%!TEX root = ../dokumentation.tex

\chapter{Projektabschluss, Fazit \& Ausblick}\label{cha:Schlussbetrachtung}
\section{Fazit}\label{sec:Fazit}

\section{Ausblick}\label{sec:Ausblick}
Ein momentaner Schwachpunkt der Datenbank ist, dass die Aussage, dass der Username schon existiert unter Umständen nicht richtig ist. Sobald der Username (z.B. test123) Teil eines schon bestehendes Usernames ist (z.B. test12345), kann die Abfrage dies nicht unterscheiden und gibt den bereits existierenden User zurück. Die Registrierung wird also verweigert, obwohl der Username noch nicht exisitert. Durch eine genauere Prüfung der bereits existierenden Usernamen könnte dies verhindert werden.

Eine denkbare Erweiterung der Anwendung ist das Speichern des Spielstandes in der Datenbank. Zur Zeit ist das Spiel nur lokal zwischengespeichert und wird gelöscht, sobald das Spiel beendet ist, die Seite neu geladen wird oder der User sich ausloggt. Durch ein Speichern des Spielstandes und des Spielfeldes könnte auch bei einem erneuten Login das alte Spiel weitergespielt werden.

Um die Anwendung intuitiver zu gestalten, können dem User beim Login und der Registrierung Bildschirmausgaben über den aktuellen Stand angezeigt werden. Diese Ausgaben könnten über eine fehlgeschlagene Anmeldung informieren. Momentan wird bei einer fehlerhaften Anmeldung nur die Seite neu geladen. 