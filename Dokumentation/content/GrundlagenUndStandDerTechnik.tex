%!TEX root = ../dokumentation.tex

\chapter{Grundlagen und Stand der Technik}\label{cha:Grundlagen}
\section{Node.JS}\label{sec:Node.JS}

\section{Websockets}\label{sec:Websockets}
<<<<<<< HEAD
Websockets sind die Grundlage für die Chat-Anwendung.
\section{Docker}\label{sec:Docker}
<<<<<<< HEAD
Docker ist eine Software ... In dieser Projektarbeit wird Docker verwendet, um das einfache deployen der Anwendungen, unabhängig von den konkreten Servern zu ermöglichen.
=======
Websockets sind die Grundlage für die Chat-Anwendung

\section{MongoDB}\label{sec:MongoDB}
In der Anwendung wird eine MongoDB zum Speichern der Zugangsdaten der Chat-Teilnehmer genutzt.
=======
Docker 
\section{MongoDB und Mongoose}\label{sec:MongoDB}
In der Anwendung wird eine MongoDB zum Speichern der Zugangsdaten der Chat-Teilnehmer genutzt.
MongoDB ist eine NoSQL-Datenbank, dies bedeutet, dass diese nicht relational sind, dokumenten orientiert
Vorteil: besser skalierbar

%https://books.google.de/books?hl=de&lr=&id=uGUKiNkKRJ0C&oi=fnd&pg=PP1&dq=mongodb&ots=h9nCKhgUrh&sig=v6FCHuD1-Sf7vlxia1SuKtF-jtQ#v=onepage&q=mongodb&f=false 

Mongoose stellt eine Objekt-Daten-Modellierungs-Bibliothek für MongoDB und Node.JS zur Verfügung.
>>>>>>> Bericht
