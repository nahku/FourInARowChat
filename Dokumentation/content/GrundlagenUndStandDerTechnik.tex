%!TEX root = ../dokumentation.tex

\chapter{Grundlagen und Stand der Technik}\label{cha:Grundlagen}
\section{Node.JS}\label{sec:Node.JS}
Node.JS ist eine JavaScript Plattform, die JavaScript außerhalb des Browsers ausführt. Node.JS wird häufig serverseitig benutzt, um Daten zu Senden/zu Empfangen.

Der Unterschied zwischen Node.JS und anderen Sprachen zur Serverprogrammierung liegt darin, dass Node.JS keinen neuen Thread für eine neue Request startet, sondern alle Request einen Single-Thread benutzten. Node.JS arbeitet asynchron und verarbeitet einen neu eintreffenden Befehl unmittelbar. Umgesetzt wird das mit nicht-blockierende I/O-Anfragen. Während des Wartens auf die Peripheriegeräte können andere Befehle ausgeführt werden. Ein weiterer Vorteil, der sich daraus erschließt, ist, dass keine Deadlocks auftreten können, weil Ressourcen nicht blockiert werden. Aus diesem Grund bietet sich Node.JS für skalierbare Netzwerkanwendungen an.

Mit Hilfe des \acf{NPM} können zusätzliche Funktionalitäten in Node genutzt werden, ähnlich zu Bibliotheken in anderen Programmiersprachen.

% https://nodejs.org/en/about/

\section{Websockets}\label{sec:Websockets}
Websockets sind die Grundlage für die Chat-Anwendung. Der Vorteil von Websockets, gegenüber z. B. einem reinen \ac{HTTP}-Protokoll, liegt darin, dass die Verbindung vom Client nur ein Mal geöffnet werden muss. Dann kann der Server dem Client Informationen senden, ohne dafür eine neue Verbindung zu benötigen. Dies ist bei Chats wichtig, da jederzeit neue Nachrichten eintreffen können, die ohne Verzögerung an den Client weitergeleitet werden sollen. Müsste der Client dafür eine neue Verbindung einrichten, so würden die Nachrichten nicht ohne Zeitverzögerungen ankommen.
\section{Docker}\label{sec:Docker}
Docker ist eine Software ... In dieser Projektarbeit wird Docker verwendet, um das einfache deployen der Anwendungen, unabhängig von den konkreten Servern zu ermöglichen. Docker ermöglicht es, Container auf Server zu deployen, und stellt dabei sicher, dass die Umgebung für die Software im Container so gleich bleibt, dass die Software genauso funktioniert, wie auf anderen Servern mit anderer Hardware. So kann Software allgemein, ohne Probleme beim Deployen auf verschiedene Server, entwickelt werden.
\section{MongoDB und Mongoose}\label{sec:MongoDB}
In der Anwendung wird eine MongoDB zum Speichern der Zugangsdaten der Chat-Teilnehmer genutzt.
MongoDB ist eine NoSQL-Datenbank, dies bedeutet, dass diese nicht relational sind, sondern Dokumenten-orientiert (S. 3)??. Ein Dokument besteht aus einer Menge an Keys und Values.
Dies bietet unter Anderem eine besser Skalierbarkeit, da die Daten besser auf verschiedene Server aufgeteilt werden können(S. 4). Es gibt keine vorgegeben Schemas, die eingehalten werden müssen. MongoDB speichert die Daten in BSON und die Rückgabeobjekte sind in JSON.

%https://books.google.de/books?hl=de&lr=&id=uGUKiNkKRJ0C&oi=fnd&pg=PP1&dq=mongodb&ots=h9nCKhgUrh&sig=v6FCHuD1-Sf7vlxia1SuKtF-jtQ#v=onepage&q=mongodb&f=false 

Mongoose stellt eine Objekt-Daten-Modellierungs-Bibliothek für MongoDB und Node.JS zur Verfügung.
Mit Mongoose kann das Datenschema der Datenbank im Code in JSON definiert werden.

%https://books.google.ca/books?hl=de&lr=&id=DxTFXO771tIC&oi=fnd&pg=PT12&dq=mongoose+mongodb&ots=MJ4ICTA1D1&sig=CMLXXEagbY1_Dn9Yt7TtTAbDBsE&redir_esc=y#v=onepage&q=mongoose%20mongodb&f=false
