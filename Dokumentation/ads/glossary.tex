%!TEX root = ../dokumentation.tex

%
% vorher in Konsole folgendes aufrufen:
%	makeglossaries 
%makeglossaries dokumentation.acn && makeglossaries dokumentation.glo

%
% Glossareintraege --> referenz, name, beschreibung
% Aufruf mit \gls{...}
%
%Befehle für Symbole



\newglossaryentry{symb:d}{
name=$\boldsymbol{d}$,
text=$d$,
description={Abstand von Sender zu Objekt [m]},
sort=symbold, type=symbolslist
}

%Normales glossar
\newglossaryentry{Glossareintrag}{name={Glossareintrag},plural={Glossareinträge},description={Ein Glossar beschreibt verschiedenste Dinge in kurzen Worten}}

\newglossaryentry{Commodity-Hardware}{name={Commodity-Hardware},description={\flqq Computer hardware that is affordable and easy to obtain. Typically it is a low-performance system that is IBM PC-compatible and is capable of running Microsoft Windows, Linux, or MS-DOS without requiring any special devices or equipment.\frqq\footcite{Beal.2015}}}

\newglossaryentry{Git}{name={Git},plural={Git},description={Git ist ein kostenloses System zur Versionskontrolle für kleine wie auch sehr große Projekte. ({\url{http://git-scm.com/}})}}

%% modsuper glossary display style
\newglossarystyle{modsuper}{%
\glossarystyle{super}%
\singlespacing
\renewcommand{\arraystretch}{1.1}
\renewenvironment{theglossary}%
    {\tablehead{}\tabletail{}%
     \begin{supertabular}{@{}lp{\glsdescwidth}}}%<----no margin
    {\end{supertabular}}%
\renewcommand{\glsgroupskip}{}%
\renewcommand*{\glossaryentryfield}[5]{%
\glsentryitem{##1}\glstarget{##1}{##2} & ##3\glspostdescription\space ##5\\[2pt]}%
\setlength{\glsdescwidth}{0.8\textwidth}


}

